% Chapter Template

\chapter{Conclusiones} % Main chapter title

\label{Chapter5} % Change X to a consecutive number; for referencing this chapter elsewhere, use \ref{ChapterX}


%----------------------------------------------------------------------------------------

%----------------------------------------------------------------------------------------
%	SECTION 1
%----------------------------------------------------------------------------------------

Este capítulo final resume el trabajo que se realizó hasta este punto, junto con su evaluación y conclusiones, y se esbozan las posibilidades de un trabajo futuro que continúe el proyecto.

\section{Trabajo realizado}
El desarrollo de CIAABOT aportó al proyecto CIAA y a la comunidad de sistemas embebidos en general con una alternativa abierta y local, de las plataformas de robótica educativa que se encuentran hoy en día.

Además, el desarrollo del IDE otorga una interfaz fácil y rápida de utilizar para programar la EDU-CIAA-NXP. Es especialmente útil para la enseñanza de la programación de embebidos como para un prototipado rápido o generación de código breve para pruebas puntuales como lecturas de sensores o manejo de actuadores.

%----------------------------------------------------------------------------------------
%	SECTION 2
%----------------------------------------------------------------------------------------
\section{Trabajo futuro}
La plataforma aún está en etapa de desarrollo. Por el lado del IDE sería bueno continuar con el desarrollo de la comunicación Firmata con la placa, para poder realizar una especie de depuración interactiva.

Además se podría incrementar el soporte para otras placas del ecosistema CIAA, y de futuros ponchos y modelos de CIAABOT.

Por la parte del desarrollo de hardware deberían armarse diferentes modelos de CIAABOTS que se especializaran en tareas puntuales, lo que implicaría el desarrollo de nuevos ponchos. Además, son necesarios los diseños en 3D para cumplir con la idea propuesta de que los CIAABOTS debería poder imprimirse y armarse sin más.

Se aprovecha esta oportunidad para invitar a cualquier interesado a participar en la continuación del desarrollo de esta plataforma abierta de robótica educativa, ya sea aportando código, diseños, sugerencias o errores descubiertos.

\section{Conclusiones personales}
Este proyecto tiene la particularidad de que se planteó con el objetivo principal de que fuera usado realmente en ambientes educativos, y se logró. El hecho de que el fruto del trabajo de un año sea puesto a prueba en el escenario para el que fue concebido y sea bienvenido de la manera que lo fue CIAABOT es realmente gratificante.

Pienso que con un poco más de desarrollo puede llegar a ser una herramienta útil para la enseñanza, y tengo la esperanza de que la comunidad de usuarios y desarrolladores de la misma crezca y logre asentar CIAABOT como un proyecto abierto, realizado en conjunto y enfocado en mejorar la calidad educativa de nuestro país.

